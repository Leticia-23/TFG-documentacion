\begin{center}
{\Large \bfseries Monitorización web en tiempo real de alertas en entornos hospitalarios}

\vspace{1cm}
{\Large \bfseries RESUMEN}

% \vspace{2.5cm}
\end{center}

Ibernex es una compañía que forma parte del Grupo Pikolin y que está especializada en el diseño, desarrollo e integración de soluciones y servicios tecnológicos destinados al sector socio-sanitario. La empresa crea soluciones para automatizar y digitalizar la atención y experiencia en residencias u hospitales. \\


Las soluciones que se han desarrollado hasta el momento están orientadas a aplicaciones de escritorio. La evolución actual de las tecnologías ha llevado a la empresa a reconsiderar este modelo y plantearse la migración de algunas de sus funcionalidades a aplicaciones web con el fin de habilitar nuevas maneras de despliegue y ejecución. Por ejemplo, mediante el uso distribuido de monitores o tablets en una residencia u hospital. A modo de prototipo inicial demostrador, se plantea este proyecto de cara a que la empresa pueda evaluar tanto la operativa de este tipo de modelos, como la complejidad de su desarrollo. En este sentido, la empresa busca también contar con un primer tutorial de desarrollo que les sirva de punto de partida a sus técnicos. \\


En este contexto, se desarrolla una aplicación web que muestra la monitorización en tiempo real de alertas. El sistema consta de dos partes. Por un lado, un frontend completo desde cero, separado de la aplicación actual de Ibernex, que recibe las alertas en tiempo real y permite visualizarlas. Por otro lado, un backend que se integra en el actual producto de Ibernex y, por lo tanto, que está sometido a sus restricciones tecnológicas y operativas.
Los principales retos a los que ha habido que enfrentarse han sido la integración del desarrollo software con un sistema ya construido y operativo; el trabajo con tecnologías distintas a las vistas en la carrera (.NET y C\#); el desarrollo de una interfaz gráfica interactiva; y la necesidad de abordar los trabajos de construcción de software teniendo que ir adaptándose a los cambios que el cliente (Ibernex) va proponiendo.

