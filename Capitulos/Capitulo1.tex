\chapter{Introducción}
\label{ch:no_lineal}

En este capítulo se presenta el contexto del trabajo, así como la motivación del problema concreto que se aborda. Se explica los objetivos y sus limitaciones, además de las herramientas de trabajo utilizadas. Por último, se explica el esquema general de la memoria del proyecto.

\section{Contexto de trabajo}

Explicar en qué trabaja la empresa, cómo trabaja, etc.

Explicar lo que se quiere implementar, primera pincelada

\begin{figure}[!h]
    \centering
    \includegraphics[width=0.8\textwidth,height=6cm]{Imagenes/Arte_abstracto}
    \caption{Un pie de foto}
    \label{fig:una_etiqueta}
\end{figure}


\section{Objetivos y limitaciones}

Explicar concretamente lo que se está implementando tanto en backend como en frontend. 

Explicar limitaciones como que en este caso la aplicación estará muy orientada a solo la funcionalidad que se pide pero que es así porque es un trabajo muy acotado ya que es una prueba para la empresa.

\section{Herramientas de trabajo}

Explicar las tecnoloías que estoy utilizando para cada una de las partes, y las librerías utilizadas.

\section{Esquema general de la memoria del proyecto}

Exponer las secciones que pertenecen a la memoria del proyecto para hacer una presentación de lo que se explica en cada una de ellas y no solo con el índice y los títulos.








