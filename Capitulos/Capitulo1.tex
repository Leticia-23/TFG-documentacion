\chapter{Introducción}
\label{ch:no_lineal}

En este capítulo se presenta el contexto del trabajo, así como la motivación del problema concreto que se aborda. Se explica los objetivos y sus limitaciones, además de las herramientas de trabajo utilizadas. Por último, se explica el esquema general de la memoria del proyecto.

\section{Contexto de trabajo}

% TODO: Explicar en qué trabaja la empresa, cómo trabaja, etc.

 Ibernex está especializada en el diseño, desarrollo e integración de soluciones y servicios tecnológicos destinados al sector socio-sanitario.
 Realizan soluciones para automatizar y digitalizar la atención y experiencia de residencias u hospitales. \\


 La compañía realiza todo el proceso de construcción del producto ya que se encarga de la fabricación del hardware y de la implementación del software de las soluciones para cada cliente.
 Los clientes de la empresa, como se ha comentado anteriormente, son residencias u hospitales que quieren digitalizar el proceso de cuidado y atención de pacientes, además de otros procesos que puedan tener según sus necesidades. \newline

% En este caso interesa la parte del software. Ibernex trabaja con una aplicación 

Duda - TODO: explicar qué es lo que tiene implementado Ibernex para sus soluciones y qué me presenta como punto de partida para desarrollar el sistema ???

% \begin{figure}[!h]
%     \centering
%     \includegraphics[width=0.8\textwidth,height=6cm]{Imagenes/Arte_abstracto}
%     \caption{Un pie de foto}
%     \label{fig:una_etiqueta}
% \end{figure}




\section{Objetivos y limitaciones}

% TODO: Explicar concretamente lo que se está implementando tanto en backend como en frontend. 

% TODO: Explicar limitaciones como que en este caso la aplicación estará muy orientada a solo la funcionalidad que se pide pero que es así porque es un trabajo muy acotado ya que es una prueba para la empresa.

El objetivo del proyecto es desarrollar un sistema de información en forma de aplicación web que muestra la monitorización en tiempo real de alertas. \newline

Estas alertas pueden ser de tres tipos:
\begin{itemize}
    \item \textbf{Tareas}: son alertas periódicas creadas para los diferentes pacientes.
    \item \textbf{Alarmas}: son alertas que se disparan una única vez.
    \item \textbf{Presencias}: son alertas que indican la presencia de un trabajador.
\end{itemize}

La construcción del sistema se basa en la implementación de una aplicación web completa realizando el desarrollo de código para un frontend completo desde cero y un backend que se realiza integrando código en la aplicación de Ibernex, aprovechando algunas de las funcionalidades existentes en la aplicación y desarrollando las nuevas funcionalidades necesarias. \newline



\section{Herramientas de trabajo}

% TODO: Explicar las tecnoloías que estoy utilizando para cada una de las partes, y las librerías utilizadas.

\section{Esquema general de la memoria del proyecto}

% TODO: Exponer las secciones que pertenecen a la memoria del proyecto para hacer una presentación de lo que se explica en cada una de ellas y no solo con el índice y los títulos.








