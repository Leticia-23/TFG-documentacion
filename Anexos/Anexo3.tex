\chapter{Decisión descarte JWT}
\label{anexo-c}

Inicialmente la empresa propuso utilizar como método de autenticación JSON Web Tokens (JWT) \cite{jwt} para cuando se realizara el inicio de sesión almacenar en local storage el token.\\

Esto se planteaba para el caso de que la arquitectura software fuese la que se planteaba inicialmente en la que se tenía un middleware ASP.NET Core como capa intermedia entre el backend, es decir la aplicación existente, y el frontend. Pero finalmente se decide junto con la empresa que la arquitectura no va a tener esa capa intermedia ya que no la consideran necesaria. (Se puede ver más información sobre las distintas arquitecturas planteadas en el \hyperref[anexo-a]{Anexo A}). \\

Por lo que la razón de esta decisión es que, al utilizar una arquitectura sin capa intermedia en la que la comunicación entre el frontend y el backend se realiza mediante websockets, no se considera necesario meter este tipo de autenticación. Sino que el inicio de sesión se realiza utilizando directamente los websockets. Esto es así también, porque el entorno de despliegue en el que estaría la aplicación web sería en la red de área local (LAN) del hospital o residencia.
