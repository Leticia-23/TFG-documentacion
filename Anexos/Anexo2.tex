\chapter{Decisión descarte SignalR}
\label{anexo-b}

% Explicar por qué no se ha utilizado SignalR respecto a las arquitecturas

SignalR \cite{signalr} es una biblioteca de código abierto que simplifica la adición de funcionalidad web en tiempo real a las aplicaciones.\\

En un principio SignalR es la opción a utilizar en la implementación propuesta por la empresa  para establecer la comunicación entre el backend y el frontend.\\

La decisión de descartar SignalR se toma en el momento en el que se observa que no es compatible con .NET Framework 4.8 (y sí lo es con ASP.NET Core) y durante la evaluación de las distintas opciones de arquitecturas software para el sistema explicadas con detalle en el \hyperref[anexo-a]{Anexo A}.\\

La razón de esta decisión es que, para utilizar SignalR, la arquitectura elegida debería haber sido la que contiene un middleware entre el frontend y backend, de tal forma que este middleware estuviese implementado con ASP.NET Core para que fuese compatible.
Con esta arquitectura la comunicación entre middleware y frontend podría haber sido totalmente compatible utilizando en React la librería \textit{microsoft/signalr} \cite{microsoft/signalr}. Pero para comunicar con el backend se debería haber utilizado en la aplicación Helpnex otra librería para websockets como \textit{websocket-sharp} \cite{websocket-sharp} que es la finalmente escogida a utilizar teniendo en cuenta la arquitectura final elegida.\\

Por lo que se toma la decisión teniendo en cuenta utilizar o no SignalR y la elección de una arquitectura más compleja o más simple y que sea la más rentable en el momento para la empresa.
