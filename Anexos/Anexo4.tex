\chapter{Tipos de alarmas}
\label{anexo-d}

Las alarmas pueden ser de los siguientes tipos:

\begin{itemize}
    \item \textbf{Normal}: es una alarma genérica, que no pertenece a ningún otro tipo.
    \item \textbf{Baño}: generada desde el tirador instalado en el cuarto de baño. Tiene más prioridad que una alarma normal porque al producirse dentro de un baño, puede ser que no se oiga o no se vea al residente desde el pasillo.
    \item \textbf{Monitor}: se produce cuando un elemento externo (por ejemplo un monitor de pulsaciones) conectado al terminal, activa una salida.
    \item \textbf{Ayuda}: se produce cuando una enfermera se ha identificado en la habitación y pulsa el botón de alarma. Significa que una enfermera está pidiendo ayuda a otra enfermera.
    \item \textbf{Médico}: se produce cuando una enfermera se ha identificado en la habitación y pulsa el botón de alarma de médico. Significa que una enfermera solicita la presencia de un médico.
    \item \textbf{Control de Errantes}: se produce cuando un paciente que no tiene permitida la salida del centro está tratando de salir sin acompañamiento.
    \item \textbf{Ayuda médico} se produce cuando un médico se ha identificado en la habitación y pulsa el botón de alarma. Significa que el médico solicita la presencia de una enfermera.
    \item \textbf{Ayuda médico y enfermera}: se produce cuando un médico y una enfermera se han identificado en la habitación y se pulsa el botón de alarma. Significa que se requiere la presencia de más enfermeras. Es la alarma con más prioridad de todas.
    \item \textbf{Accesos}: se genera cuando un acceso controlado por Helpnex se abre se forma incorrecta o sin permiso.
    \item \textbf{Fuera de centro}: se genera cuando un tag de localización ha dejado de emitir (y de localizarse) por un tiempo superior a uno determinado. Sirve para detectar fugas y para cuando el tag se queda sin batería.
\end{itemize}